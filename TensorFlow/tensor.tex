\documentclass[11pt]{article}
\usepackage{listings}
%Gummi|065|=)
\title{\textbf{Tensorflow}}
\author{Carlos Uscanga\\
		Juan Carlos Olivier\\
		Angel Verga}
\date{17 de Mayo de 2016}
\begin{document}

\maketitle

\section{Acerca de Tensorflow}
TensorFlow es una biblioteca de software de código abierto para el cálculo numérico usando diagramas de flujo de datos. Los nodos del gráfico representan operaciones matemáticas, mientras que los bordes del gráfico representan los arreglos de datos multidimensionales (tensores) comunicados entre ellos. La arquitectura flexible le permite implementar el cálculo de una o más CPU o GPU en una computadora de escritorio, servidor o dispositivo móvil con una sola API. TensorFlow fue originalmente desarrollado por investigadores e ingenieros que trabajan en el equipo de cerebro Google dentro de la organización de investigación de la máquina de Inteligencia de Google a los efectos de llevar a cabo el aprendizaje de máquina y profunda investigación de redes neuronales, pero el sistema es lo suficientemente generales como para ser aplicable en una amplia variedad de otros dominios bien.

\section{Descarga}
\subsection{Requerimientos}
La API TensorFlow Python soporta Python 2.7 y Python 3.3+.

La versión de la GPU (sólo Linux) funciona mejor con Cuda Toolkit 7.5 y v4 cuDNN. otras versiones son compatibles (Cuda conjunto de herramientas> = 7.0 y 6.5 cuDNN (v2), 7.0 (v3), v5) sólo cuando la instalación de fuentes. Por favor, vea la instalación Cuda para más detalles.

\begin{lstlisting}[language=bash]
# Ubuntu/Linux 64-bit
$ sudo apt-get install python-pip python-dev

# Mac OS X
$ sudo easy_install pip

2. Instalación de tensorflow

# Ubuntu/Linux 64-bit, CPU only:
$ sudo pip install --upgrade https://storage.googleapis.com/tensorflow/linux/cpu/tensorflow-0.8.0-cp27-none-linux_x86_64.whl

# Ubuntu/Linux 64-bit, GPU enabled. Requires CUDA toolkit 7.5 and CuDNN v4.  For
# other versions, see "Install from sources" below.
$ sudo pip install --upgrade https://storage.googleapis.com/tensorflow/linux/gpu/tensorflow-0.8.0-cp27-none-linux_x86_64.whl

# Mac OS X, CPU only:
$ sudo easy_install --upgrade six
$ sudo pip install --upgrade https://storage.googleapis.com/tensorflow/mac/tensorflow-0.8.0-py2-none-any.whl

\end{lstlisting}

\begin{lstlisting}[language=python]
//3. Hola mundo en Tensorflow
$ python
...
>>> import tensorflow as tf
>>> hello = tf.constant('Hello, TensorFlow!')
>>> sess = tf.Session()
>>> print(sess.run(hello))
Hello, TensorFlow!
>>> a = tf.constant(10)
>>> b = tf.constant(32)
>>> print(sess.run(a + b))
42
>>>
\end{lstlisting}
\newpage
\section{laboratorio}
Nosotros decidimos realizar el laboratorio del reconocimiento de imágenes.

A pesar de que los procesos de cognición visual son una tarea sencilla para los seres humanos, esto es más complicado de resolver mediante el uso de computadoras. En los últimos años, el campo de “machine learning” ha hecho un considerable progreso en lo que a superar estas dificultades se refiere; particularmente se ha descubierto que un modelo llamado red neuronal convolucional puede realizar un razonable desempeño en tareas de reconocimiento visual difíciles, igualando o inclusive excediendo el desempeño humano en algunas áreas.
Investigadores han demostrado un progreso continuo en visión computarizada. Modelos sucesivos continúan mostrando mejoras, cada vez alcanzando nuevos resultados, como lo son: QuocNet, AlexNet, Inception(GoogLeNet), BN-Inception-v2. Investigadores tanto internos como externos a Google han publicado artículos describiendo todos estos modelos, sin embargo, los resultados todavía son difíciles de reproducir.
Inception-v3 es el modelo más reciente en cuanto a reconocimiento de imagen. Una de las tareas estándar que puede realizar consiste en tratar de clasificar imágenes completas en 1000 clases. Para comparar modelos, se examina qué tan seguido un modelo falla al predecir la respuesta correcta dentro de sus 5 mejores conjeturas. Esto se puede hacer mediante el uso de los lenguajes de programación Python o C++.
Esto se aplica a una imagen mediante la función ReadTensorFromImageFile().

\begin{lstlisting}
Status ReadTensorFromImageFile(
	string file_name, const int input_height,
    	const int input_width, const float input_mean,
    	const float input_std,
        std::vector<Tensor>* out_tensors) {
  tensorflow::GraphDefBuilder b;

\end{lstlisting}

Se empieza creando un GraphDefBuilder, que es un objeto que podemos usar para especificar el modelo a utilizar.

\begin{lstlisting}
string input_name = "file_reader";
  string output_name = "normalized";
  tensorflow::Node* file_reader =
      tensorflow::ops::ReadFile(
      	tensorflow::ops::Const(file_name, b.opts()),
        b.opts().WithName(input_name));
\end{lstlisting}
Luego creamos nodos para el modelo que queremos cargar, y escalar los valores de pixeles para obtener el resultado que el modelo principal espera como entrada. El primer nodo creado es un valor constante que mantiene un tensor con el nombre del archivo de la imagen que queremos cargar. Posteriormente pasa como la primera entrada del ReadFile.

\begin{lstlisting}
const int wanted_channels = 3;
  tensorflow::Node* image_reader;
  if (tensorflow::StringPiece(file_name).ends_with(".png")) {
    image_reader = tensorflow::ops::DecodePng(
        file_reader,
        b.opts().WithAttr("channels", 
        wanted_channels).WithName("png_reader"));
  } else {
    image_reader = tensorflow::ops::DecodeJpeg(
        file_reader,
        b.opts().WithAttr("channels",
         wanted_channels).WithName("jpeg_reader"));
  }
  tensorflow::Node* 
  float_caster = tensorflow::ops::Cast(
      image_reader, tensorflow::DT_FLOAT,
       b.opts().WithName("float_caster"));
  tensorflow::Node*
   dims_expander = tensorflow::ops::ExpandDims(
      float_caster, 
      tensorflow::ops::Const(0, b.opts()), b.opts());
  tensorflow::Node* resized = 
  tensorflow::ops::ResizeBilinear(
      dims_expander, tensorflow::ops
      ::Const({input_height, input_width},
              b.opts().WithName("size")),
      b.opts());
  tensorflow::ops::Div(
      tensorflow::ops::Sub(
          resized, tensorflow::ops::
          Const({input_mean}, b.opts()), b.opts()),
      tensorflow::ops::Const({input_std}, b.opts()),
      b.opts().WithName(output_name));

\end{lstlisting}
Luego seguimos añadiendo más nodos para decodificar los datos del archivo como una imagen.

\begin{lstlisting}
tensorflow::GraphDef graph;
  TF_RETURN_IF_ERROR(b.ToGraphDef(&graph));
Al final, main() une todas las llamadas del sistema.
int main(int argc, char* argv[]) {
  tensorflow::port::InitMain(argv[0], &argc, &argv);
  Status s = tensorflow::ParseCommandLineFlags(&argc, argv);
  if (!s.ok()) {
    LOG(ERROR) << 
    "Error parsing command line flags: " << s.ToString();
    return -1;
  }
  std::unique_ptr<tensorflow::Session> session;
  string graph_path = tensorflow::io::
  JoinPath(FLAGS_root_dir, FLAGS_graph);
  Status load_graph_status = LoadGraph(graph_path, &session);
  if (!load_graph_status.ok()) {
    LOG(ERROR) << load_graph_status;
    return -1;
  }

\end{lstlisting}

Cargamos el grafo principal.
\begin{lstlisting}

std::vector<Tensor> resized_tensors;
  string image_path = tensorflow::io::
  JoinPath(FLAGS_root_dir, FLAGS_image);
  Status read_tensor_status = ReadTensorFromImageFile(
      image_path, FLAGS_input_height, 
      FLAGS_input_width, FLAGS_input_mean,
      FLAGS_input_std, &resized_tensors);
  if (!read_tensor_status.ok()) {
    LOG(ERROR) << read_tensor_status;
    return -1;
  }
  const Tensor& resized_tensor = resized_tensors[0];
\end{lstlisting}

Cargamos y procesamos la imagen de entrada.

\begin{lstlisting}
std::vector<Tensor> outputs;
  Status run_status = session->
  Run({{FLAGS_input_layer, resized_tensor}},
              {FLAGS_output_layer}, {}, &outputs);
  if (!run_status.ok()) {
    LOG(ERROR) << 
    "Running model failed: " << run_status;
    return -1;
  }

\end{lstlisting}
Corremos el grafo con la imagen como una entrada.

\begin{lstlisting}
    bool expected_matches;
    Status check_status = 
     CheckTopLabel(outputs, 866, &expected_matches);
    if (!check_status.ok()) {
      LOG(ERROR) << 
      "Running check failed: " << check_status;
      return -1;
    }
    if (!expected_matches) {
      LOG(ERROR) << "Self-test failed!";
      return -1;

\end{lstlisting}

Finalmente se imprimen las etiquetas encontradas.
\begin{lstlisting}
if (!print_status.ok()) {
    LOG(ERROR) << 
    "Running print failed: " << print_status;
    return -1;
  }

\end{lstlisting}

\newpage

\section{Codigo}
\lstinputlisting[language=Python]{classify_image.py}
\newpage
\section{Conclusiones}
La tecnología de reconocimiento visual y visión computarizada, pese a estar en fases de desarrollo iniciales, es ciertamente alcanzable. En base a esta información, se puede concluir que este es un campo aún por explorar, con muchas aplicaciones y beneficios potenciales, los cuales pueden ser implementados en diferentes ámbitos y campos, como lo son reconocimiento audio-visual, inspección óptica automatizada, anotación automática de imágenes, reconocimiento de objetivos automático, reconocimiento digital de huellas, reconocimiento de gestos faciales, modelado basado en imágenes, entre otros. Todo esto abre nuevas posibilidades que pueden ser incorporadas a las tecnologías actuales, de ser desarrolladas adecuadamente.

\end{document}
